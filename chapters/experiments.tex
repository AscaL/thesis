%TODO: obviously rewrite this
In this section I will describe the stack used for this thesis and the experiments done to achieve our results: I will start by reviewing the database (Par. \ref{rldb}), the configuration and parameters, the experiments results, and finally the conclusion.

\section{Building Blocks}
To realize this work we used the following tools and libraries:
\begin{itemize}
	\item OpenFace: the library to train the models for face detection, landmark detection, feature extraction and action unit recognition.
	\item R + R-Studio: the environment where I implemented the code, with some important libraries, such as:
	\begin{itemize}
		\item e1071: library for SVM classification.
		\item Tidyverse: General library for %todo
		\item Random Forest: library to perform Random Forest and calculate variable importance.
		\item Corrplot: library for calculating and visualizing correlations.
	\end{itemize}
\end{itemize}

\clearpage

%TODO: SHORT review (table pherhaps) of face db, maybe not in this section
\section{Real Life Trial Database} \label{rldb}
This section is about the database we used to perform our experiments: it comes from the work done for the paper "Deception Detection using Real-life Trial Data" \cite{Perez-Rosas:2015:DDU:2818346.2820758}.

\begin{figure}[H]
	\centering
	\includegraphics[width=1\textwidth]{trial_images}
	\caption{Examples of images from the dataset videos. \cite{Perez-Rosas:2015:DDU:2818346.2820758}.}
	\label{fig:trial_images}
\end{figure}


The dataset is gathered from real-life trial videos available on YouTube and other public websites. The dataset also contains statements made by exonerees after exoneration, and some statements from defendants during crime-related TV episodes.

The first step to collecting the dataset was to identify public multimedia sources where the recordings of the trials were available, and deceptive and truthful behavior could be observed and verified.\\
The videos are of trial recordings where the defendant or witness in the video can be clearly identified, the face is visible enough during most of the clip duration, and the visual quality should be good enough to accurately see the facial expressions (Fig. \ref{fig:trial_images}).\\
There are three outcomes for the trials that were considered to label the videos as deceptive or truthful: guilty, non-guilty, and exoneration. \\
For the guilty verdicts the deceptive clips are taken from the defendant in the trial, while the truthful clips are gathered from the witnesses. There are also instances where the deceptive videos are of suspects denying a committed crime, and truthful ones are from the same person answering questions that where verified by the police as truthful.

In regards to the witnesses, if the testimony is verified by a police officer they are labeled as true. \\ Testimonies that help the guilty party are labeled as false. Exoneration (reversal of the sentence) testimonies are regarded as truthful.

The original dataset consists of 121 videos, 61 of which are deceptive and 60 truthful. \\
The average length of the videos is 28.0 seconds. The average video length for deceptive videos is 27.7 seconds, while the one for truthful videos is 28.3 seconds. \\
The data consists of 58 total subject, 22 females and 36 males, with ages between 16 and 60 years.

This dataset was annotated following the MUMIN coding scheme for hand movement and facial displays. While annotating, the annotators were able to chose only one label per gesture, for every clip.

Fig. \ref{fig:rldb_distrib} shows the frequency counts associated with the gestures considered during the annotation.

\begin{figure}[H]
	\centering
	\includegraphics[width=1\textwidth]{rldb_distrib}
	\caption{Examples of images from the dataset videos. \cite{Perez-Rosas:2015:DDU:2818346.2820758}.}
	\label{fig:rldb_distrib}
\end{figure}

%TODO: modification made by me to the DB 
%TODO: rewrite tentative atm

We modified this database in the following way:
\begin{itemize}
	\item We cut or removed videos where the where the face was covered, hidden or in very difficult to see.
	\item We also cut parts of video that had multiple subjects in the scene because Action Unit extraction works on one person only.
	\item Since it is necessary to see the person to perform AU extraction, we also avoided some videos in which the subject was not visible while talking (the interlocutor was shown).
\end{itemize}

With deceptive videos going from \#1 to \#61 and truthful videos going from \#1 to \#60 we cut and removed the following videos from the original dataset, since they showed signs of the problems just described:

\begin{itemize}
	\item \textbf{Deceptive}:
	\begin{itemize}
		\item CUT: Video numbers 46, 47, 48, 49, 52, 54, 56.
		\item REMOVED: Video numbers 50, 53,  55.
	\end{itemize}
	\item \textbf{Truthful}:
	\begin{itemize}
		\item CUT: Video number 7, 28, 43.
		\item REMOVED: Video numbers 12, 31. 
	\end{itemize}
\end{itemize}


We also performed a division of subjects in the training and test set "by hand" to train the classifiers in a way that it wouldn't adapt the specific people, since there are many videos with the same subject. In fact, the same person never appears both in the training and in the test set.

%todo: stats of db, #of frames ecc

\clearpage

\section{Data Analysis}

\subsection{Data Comparison}
The first thing I did was to compare the extracted deceptive (Fig \ref{fig:au_occ_dec}) and truthful (Fig. \ref{fig:au_occ_truth}) AUs occurrences (presence, not intensity) from the training set, which is shown in Fig. \ref{fig:au_occ_comp}.

\begin{figure}[H]
	\centering
	\includegraphics[width=1\textwidth]{images/au_occ_dec}
	\caption{Average AU Occurrences for the deceptive training set.}
	\label{fig:au_occ_dec}
\end{figure}

\begin{figure}[H]
	\centering
	\includegraphics[width=1\textwidth]{images/au_occ_truth}
	\caption{Average AU Occurrences for the truthful training set.}
	\label{fig:au_occ_truth}
\end{figure}

\begin{figure}[H]
	\centering
	\includegraphics[width=1\textwidth]{images/au_occ_comp}
	\caption{Comparison of AU Occurrences for the training set.}
	\label{fig:au_occ_comp}
\end{figure}

This comparison shows interesting differences between truthful and deceptive occurrences for AU04 (Brow Lowerer), AU05 (Upper Lid Raiser), AU10 (Upper Lip Raiser), AU12 (Lip Corner Puller) and AU14 (Dimpler). %"suggesting there could be a pattern"?

\subsection{Correlation}
Variables correlation is shown in figure \ref{fig:correlation_matrix}. When the AUs label ends with "\_r" it indicate intensity and "\_c" means presence.

\begin{figure}[H]
	\centering
	\includegraphics[width=1\textwidth]{images/correlation_matrix}
	\caption{Correlation between variables for both presence and intensity in the training set.}
	\label{fig:correlation_matrix}
\end{figure}

%todo: positive corr with, neg corr with...
which shows some obvious correlations between presence and intensity of the same AU, and significant correlation between AU10\_r, AU12\_r, AU12\_c, AU07\_r, AU17\_r, AU06\_r, AU04\_c, AU02\_r, AU01\_r, AU25\_r, AU45\_c.

\subsection{GLM}
Another quick experiment I made was with the Generalized Linear Model in R, which translates into Logistic Regression (Par. \ref{logreg}), at least to have an idea of the resulting p-values and variable ranking.




\subsection{Random Forest}



\section{LDA}

\section{QDA}

\section{SVM}