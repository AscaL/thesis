In this chapter we will give a brief introduction of Machine Learning (\ref{ml}), explaining Classification (\ref{classification}), Regression (\ref{regression}) and Supervised learning (\ref{supervised}), proceeding then to explain different techniques such as Random Forests (\ref{random_forest}) and SVM (\ref{svm}), which is the most important technique used for this thesis.

\section{Introduction} \label{ml}

\begin{figure}[H]
	\centering
	\includegraphics[width=1\textwidth]{ml_troll}
	\label{fig:ml_troll}
\end{figure}

Machine Learning is a subfield of Artificial Intelligence that uses statistical techniques to provide computers with the ability to progressively improve performance on different tasks using data, while not being explicitly programmed \cite{wiki:ml}.

The applications for machine learning are huge and diverse, and range from character recognition to email filtering, with lots of applications in computer vision, such as image recognition and classification.

Machine learning also focuses on making prediction on data, by utilizing techniques taken from statistics and mathematical optimization. This has many applications, ranging from health care by predicting risk factors for diseases or gaining insights for prevention, to sports or politics where actual results of games or elections can be predicted.

The field of machine learning is subdivided in two broad categories (Fig. \ref{fig:ml_mldiv}), supervised learning (par. \ref{supervised}) and unsupervised learning (par. \ref{unsupervised}) based on whether the data is labeled or not.

\begin{figure}[H]
	\centering
	\includegraphics[width=1\textwidth]{ml_mldiv}
	\caption{Machine Learning Subfields \cite{ml_mldiv}}
	\label{fig:ml_mldiv}
\end{figure}

Another subdivision of the field of machine learning is based on the kind of output that the users wish to obtain, and the categories of said output are:
\begin{itemize}[noitemsep]
	\item \textbf{Classification} (par. \ref{classification}) for predicting a discrete class label for a new example data.
	\item \textbf{Regression} (par. \ref{regression}) to predict a continuous quantity.
	\item \textbf{Clustering} to group similar but unlabeled data.
\end{itemize}

\subsection{Supervised Learning} \label{supervised}
When using supervised learning we want to learn a function that maps an input to an output, based on example input-output pairs \cite{ai_sup}. \\
This means that we need labeled training data, consisting of a vector of input objects and an output value. The objective is to correctly classify the new data based on the previously analyzed training pairs.

The general steps to solve a supervised learning problem are (fig. \ref{fig:sup_wf}):
\begin{enumerate}[noitemsep]
	\item Understand what kind of training example to use.
	\item Gathering a training set.
	\item Model the training set to be fed as input to the algorithm by choosing which features to use and how to represent the data.
	\item Choose what kind of algorithm can best train the model.
	\item Run the algorithm and evaluate the resulting accuracy on the test set.
\end{enumerate}

\begin{figure}[H]
	\centering
	\includegraphics[width=1\textwidth]{sup_wf}
	\caption{Supervised Learning Workflow \cite{sup_wf}}
	\label{fig:sup_wf}
\end{figure}

There are important considerations to make when using a supervised learning approach:
\begin{itemize}
	\item \textbf{Bias-Variance Tradeoff} \cite{biasvar}: \\	
	Suppose we have many different (but equally good) training sets: \\
	an algorithm is biased for input $x$ if, when trained on each of these data sets, it is consistently inaccurate at predicting the correct output for $x$. \\
	A high bias indicates that the data points tend to be very close to the mean, and to each other.
	
	A learning algorithm has high variance for a particular input $x$ if it predicts output values that are correct on average but inconsistent when trained on different training sets. \\
	Basically, a high variance indicates that the data points are very spread out from the mean, and from one another (Fig. \ref{fig:biasvar}).
	
	The prediction error of a classifier is related to the sum of bias and variance, so generally there is a tradeoff between them. Low bias means that it fits the new data well, but if the bias is too low it will fit each training set differently and so result in high variance.
	
	\begin{figure}[H]
		\centering
		\includegraphics[width=0.75\textwidth]{biasvar}
		\caption{Example of Bias-Variance Tradeoff \cite{biasvarTradeoff}}
		\label{fig:biasvar}
	\end{figure}	

	\item \textbf{Dimensionality of Input} \cite{wiki:dim_red}:\\ 
	When the input feature vectors are dimensionally very big there could be problems in learning the function, even if not all features contribute significantly to it. This happens because the data depends on too many variables and this could cause high variance. \\
	To avoid this, it is important to reduce the number of features through manual removal or using feature selection algorithms such as Principal Component Analysis (PCA). This usually improves the accuracy of the classifier.
	
	\item \textbf{Overfitting and Underfitting} \cite{overfit}: \\	
	Overfitting happens when the algorithm adapts too much on the training data and is no longer able to make accurate predictions on the test data (Fig. \ref{fig:underfitting}). This usually happens when there is an excessive number of parameters than what can be justified by the data \cite{camb_over}.
	
	Underfitting is the opposite: a model is not able to correctly capture the structure of the data, for example when fitting non linear data with a linear model. \\
	A common way to avoid overfitting is to resample the data using different techniques, commonly k-fold cross validation or leave-one-out. Other methods include feature removal, early stopping and regularization.
	
	\begin{figure}[H]
		\centering
		\includegraphics[width=1\textwidth]{underfitting}
		\caption{Example of overfitting and underfitting \cite{underfitting}}
		\label{fig:underfitting}
	\end{figure}
\end{itemize}

There are many algorithms used to perform supervised learning tasks, the most commonly used are:

\begin{itemize}[noitemsep, topsep = -5pt]
	\item Linear Regression (\ref{regression})
	\item Logistic Regression (\ref{logreg})
	\item Naive Bayes
	\item Linear Discriminant Analysis
	\item Decision Trees
	\item k-Nearest Neighbor
	\item Neural Networks
	\item Support Vector Machines (\ref{svm})
\end{itemize}

\subsection{Unsupervised Learning} \label{unsupervised}
Unsupervised learning is the subfield of Machine Learning tasked with inferring a function from the analysis of unlabeled (not classified) data. Being unclassified there is also a difficulty in evaluating the accuracy of the model.\\
Usually items are grouped by some measure of similarity, like for example in k-means clustering (Fig. \ref{fig:clustering}).

\begin{figure}[H]
	\centering
	\includegraphics[width=1\textwidth]{clustering}
	\caption{Example of Unclustered and Clustered data \cite{kmeans}}
	\label{fig:clustering}
\end{figure}	

These are the most widely used unsupervised learning algorithms:

\begin{itemize}[noitemsep]
	\item Clustering
	\begin{itemize}[noitemsep]
		\item k-means
		\item mixture models
		\item hierarchical clustering
	\end{itemize}
	\item Anomaly detection
	\item Neural Networks
	\begin{itemize}[noitemsep]
		\item Autoencoders
		\item Deep Belief Nets
		\item Hebbian Learning
		\item Generative Adversarial Networks
		\item Self-organizing map
	\end{itemize}
	\item Expectation–maximization algorithm (EM)
	\item Method of moments
	\item Blind signal separation techniques
	\begin{itemize}[noitemsep]
		\item Principal component analysis,
		\item Independent component analysis,
		\item Non-negative matrix factorization
		\item Singular value decomposition.
	\end{itemize}
\end{itemize}

\subsection{Classification} \label{classification}
In machine learning, classification is the problem of identifying in which of a set of categories a new observation belongs, based on a training set of data containing observations whose category is known in advance. A common example could be classifying an email as spam or not \cite{wiki:classification}.

Classification is considered an instance of supervised learning, based on instances where a training set is available. As for unsupervised learning, classification would be clustering since it groups data into categories based on similarity, but without knowing the label of the data.

The observations, called features or explanatory variables, take different types based on the value. They can be categorical, numerical or ordinal, or be compared by similarity between previous observations using some kind of distance function.\\
The observations representing the categories to be predicted are called explanatory variables (or regressors, or independent variables).

The classifier is the algorithm that implements the classification, or a function that maps input data to a category.

\begin{figure}[H]
	\centering
	\includegraphics[width=0.75\textwidth]{classification}
	\caption{Example of data division by a classification algorithm}
	\label{fig:classification}
\end{figure}	

Many classification algorithm, such as SVM (\ref{svm}), logistic regression \ref{logreg}, LDA (Linear Discriminant Analysis) or perceptron, can be described using a linear function assigning a score to each category \textit{c}, by doing the dot product of the feature vector of an instance with a vector of weights, thus combining them. The predicted category will be the one with the highest score. This function is called linear predictor function and has this general formula:

\begin{equation}
score(X_i, c) = \beta_c \cdot X_i
\end{equation}

where $X_i$ is the feature vector of instance \textit{i} and $\beta_c$ is the vector of weights of category \textit{c}. This kind of algorithms are known as linear classifiers.

\subsection{Regression Analysis} \label{regression}
%TODO: check if it's ok
\begin{figure}[H]
	\centering
	\includegraphics[width=0.75\textwidth]{reg_analysis}
	\caption{Regression Analysis can solve this! \cite{reg_analisys}}
	\label{fig:reg_analysis}
\end{figure}

Regression analysis is used in statistics and machine learning to estimate the relationships among variables. \\
More specifically, it focuses on the relationship between one dependent variable and more independent variables (also called predictors), and generally is used to estimate the average value of the dependent variable when the independent variables are fixed, by calculating a regression function.

By understanding the relationship between dependent and independent variables it's also possible to perform predictions on future data based on new inputs, or on changes to the old ones \cite{wiki:reg_an}.

In regression analysis, it is also of interest to characterize the variation of the dependent variable around the prediction of the regression function using a probability distribution.

\begin{figure}[H]
	\centering
	\includegraphics[width=0.85\textwidth]{reg_plot}
	\caption{Example of Regression Analysis}
	\label{fig:reg_plot}
\end{figure}	

\subsubsection*{Regression Model} 
A general regression model uses the following variables and parameters:
\begin{itemize}[noitemsep]
	\item The unknown parameters, $\beta$, that represents either a scalar or a vector.
	\item The independent variables, $X$. 
	\item The dependent variable, $Y$.
\end{itemize}

A regression model relates $Y$ to a function of $X$ and $\beta$ \cite{wiki:reg_an}.
\begin{equation}
	Y \approx f(X,\beta)
\end{equation}

This is usually formalized as:
\begin{equation}
	E(Y|X) = f(X,\beta)
\end{equation}

%TODO: riguardare anche questo, troppo wiki
If $\beta$ is of length $k$ and the number of observed data points is enough ($N > k$), then it's possible to estimate a unique value for $\beta$ that best fits the data.

If this is the case, regression analysis provides the means to find a solution for unknown parameters of $\beta$ to, for example, apply the method of least squares.

To apply regression the data must abide by some assumptions, generally \cite{wiki:lin_reg}:
\begin{itemize}[noitemsep, topsep = -5pt]
	\item The sample is representative of the population for the inference prediction.
	\item The error is a random variable with a mean of zero.
	\item The independent variables are measured with no error.
	\item The independent variables (predictors) are linearly independent.
	\item The errors are uncorrelated.
	\item The variance of the error is constant across observations.
\end{itemize}

\subsection{Linear Regression} \label{lin_reg}
Linear regression is the most basic case of Regression Analysis, and it is a useful tool for predicting a quantitative response. Also, most of the newer approaches to regression are often a generalization or extension of linear regression.

Linear regression is a linear approach to modeling the relationship between a dependent variable and one or more independent variables. The simplest case of linear regression (LR) is when there is only one independent variable, and is called simple linear regression \cite{wiki:lin_reg} and has this form:
\begin{equation}
	y_i \approx \beta_0 + \beta_1x_i + \varepsilon_i
\end{equation}

where $\varepsilon_i$ is the error for the i\textit{th} observation. The error is a catch-all for what we miss with this simple model, since it's very probable that the true relationship is not linear, and that there may be other variables that influence $y$ \cite{ISLR}.

The aim of LR is to estimate the $\beta$ coefficients to make predictions. To do so, we utilize of the training dataset to produce estimates $\widehat{\beta_0}$ and $\widehat{\beta_1}$ for the model coefficients. We can then make predictions by computing:
\begin{equation}
	\widehat{y_i} \approx \widehat{\beta_0} + {\widehat{\beta_1}x_i}
\end{equation}

$e_i = y_i - \widehat{y_i}$ is the difference between the true value and the prediction for an observation, and it is called residual (Fig. \ref{fig:reg_error}).

\begin{figure}[H]
	\centering
	\includegraphics[width=0.75\textwidth]{reg_error}
	\caption{The fit is found by minimizing the sum of squared errors. Each gray line segment represents an error, and the prediction makes a compromise by averaging their square \cite{ISLR}}
	\label{fig:reg_error}
\end{figure}

The most common method for estimation is called least squares. This method obtains parameter estimates that minimize the sum of squared residuals (SSR):
\begin{equation}
	SSR=\sum _{i=1}^{n}e_{i}^{2}
\end{equation}

The minimizers are
\begin{equation}
	\widehat{\beta_{1}} = \frac{\sum(x_i - \bar{x})(y_i - \bar{y})}{\sum(x_i - \bar{x})^2}
\end{equation}

\begin{equation}
	\widehat{\beta_{0}} = \bar{y} - \widehat{\beta_{1}} \bar {x}
\end{equation}

where $\bar{x}$ and $\bar{y}$ are the mean of $x$ and $y$.

If we assume that the error term has constant variance, the estimate of the variance of the error is given by:

\begin{equation}
	\widehat{\sigma}_{\varepsilon }^{2} = \frac{SSR}{n-2}
\end{equation}

And is called mean square error (MSE) of the regression. The denominator is the sample size reduced by the number of model parameters estimated from the same data, $(n-p)$ for $p$ regressors or  $(n-p-1)$ if an intercept is used \cite{MSE}. In the case of simple linear regression $p=1$, so the denominator is $n-2$.

We can then estimate the standard errors for the parameters, that tell us the average amount that the estimate differs from the actual value.
\begin{equation}
	{\widehat{\sigma}_{\beta _{1}} = {\widehat{\sigma}_{\varepsilon}{\sqrt{\frac{1}{\sum(x_{i} - {\bar {x}})^{2}}}}}}
\end{equation}
\begin{equation}
	{\widehat{\sigma}_{\beta _{0}} = {\widehat{\sigma}_{\varepsilon}{\sqrt{\frac{1}{n} + \frac{\bar{x}^2}{\sum(x_{i} - {\bar {x}})^{2}}}}}}
\end{equation}

Standard errors can be used to compute confidence intervals. A 95\% confidence interval is defined as a range of values such that with 95\% probability, the range will contain the true unknown value of the parameter.

The general multiple regression model, where there are $p$ independent variables, follows this formula:
\begin{equation}
	y_{i}=\beta_{1}x_{i1} + \beta_{2}x_{i2} + \cdots + \beta_{p}x_{ip} + \varepsilon_{i}
\end{equation}

where $x_{ij}$ is the i-th observation on the j-th independent variable.

\clearpage

\section{Logistic Regression} \label{logreg}
In regression analysis, logistic regression is a method for estimating the parameters of a logistic model. A logistic model is one where the log-odds of the probability of an event is a linear combination of independent or predictor variables. \\
The two possible dependent variable's values are often labeled as "0" and "1", "true" or "false", "pass" or "fail" etc, and they represent binary outcomes (Fig. \ref{fig:logistic_exam}). \\
It's possible to generalize the binary logistic regression model to more than two levels (outcomes) of the dependent variable \cite{wiki:logisticreg}.

The binary logistic model works by estimating the probability of a binary response based on one or more predictor variables.\\
By using logistic regression is possible to say how much that the presence of a risk factor increases the odds of a given outcome. \\
The model itself simply calculates the probability of output in terms of input, and it is not specifically a classifier, but it can be used to classify data by choosing a threshold value, for example if the probability is $\ge 50\%$ the class is "1" otherwise is "0".

\begin{figure}[H]
	\centering
	\includegraphics[width=0.85\textwidth]{exam_pass_logistic_curve}
	\caption{Graph of a logistic regression curve showing probability of passing an exam versus hours studying \cite{wiki:logisticreg}}
	\label{fig:logistic_exam}
\end{figure}

\subsubsection{Logistic Function}
The logistic function to estimate the probabilities is a sigmoid function (exemplified in fig. \ref{fig:sigmoid}). \\
A sigmoid function takes an input from the real numbers and outputs a value between $0$ and $1$.

\begin{figure}[H]
	\centering
	\includegraphics[width=0.75\textwidth]{sigmoid}
	\caption{Example of a Sigmoid shaped function}
	\label{fig:sigmoid}
\end{figure}

We want to model the relationship between $p(X) = Pr(Y = 1|X)$ and $X$. To model this relationship we use the a logistic function:

\begin{equation} \label{eq:logr1}
p(X) = \frac{e^{\beta_{0}} + e^{\beta_{1}X}}{1+ e^{\beta_{0}} + e^{\beta_{1}X}} = \frac{1}{1 + e^{-({\beta_{0} + \beta_{1}X})}}
\end{equation} 

Equation \ref{eq:logr1} can be manipulated into: 

\begin{equation} \label{eq:logr2}
\frac{p(X)}{1 - p(X)} = e^{\beta_{0}} + e^{\beta_{1}X}
\end{equation} 

$\frac{p(X)}{1 - p(X)}$ is called odds, and can take any value between $0$ and $\infty$.

By doing the logarithm of \ref{eq:logr2} we get:

\begin{equation} \label{eq:logr3}
\log{\frac{p(X)}{1 - p(X)}} = \beta_{0} + \beta_{1}X
\end{equation} 

The coefficients $\beta_{0}$ and $\beta_{1}$ in equation \ref{eq:logr1} are unknown, and must be estimated based on the available training data. \\
To fit the model of equation \ref{eq:logr3}, the maximum likelihood method is commonly used. \\
Intuitively we want to estimates $\beta_{0}$ and $\beta_{1}$ such that the predicted probability ${\hat{p}}(x_i)$ for a specific case corresponds as closely as possible to the observed class for that case.\\
This can be achieved by using the likelihood function:

\begin{equation} \label{eq:logr4}
\ell(\beta_{0}, \beta_{1}) = \prod_{i:y_i = 1}p(x_i) \prod_{i':{y_{i'}} = 0} (1- p(x_{i'}))
\end{equation} 

The estimates $\hat{\beta_{0}}$ and $\hat{\beta_{1}}$ are chosen to maximize this likelihood function.\\
Maximum likelihood is a very general approach that is used to fit many of the non-linear models. For example in linear regression (par. \ref{lin_reg}), the least squares approach is a special case of maximum likelihood.

Once the coefficient have been estimated is pretty simple to compute the probability:
\begin{equation}
\hat{p}(X) = \frac{e^{\hat{\beta_{0}}} + e^{\hat{\beta_{1}}X}}{1+ e^{\hat{\beta_{0}}} + e^{\hat{\beta_{1}}X}}
\end{equation}

\clearpage

\section{Random Forest} \label{random_forest}

\begin{figure}[H]
	\centering
	\includegraphics[width=1\textwidth]{RFClass}
	\caption{Example of Random Forest Classification \cite{medium:RF}}
	\label{fig:RFClass}
\end{figure}

Random forests are an ensemble learning method used mainly for classification and regression. They work by building many decision trees while in the training phase, and then outputting the class that is the mode of the classes for classification, or the mean prediction of the individual trees for regression \cite{wiki:randomforest}\cite{RDF}.

Random Forest (RF) is preferred to decision trees because decision trees suffer from high variance. By using tree bagging, which is a commonly used technique to reduce variance, we can correct decision trees' habit of overfitting to their training set \cite{ESL}.

\subsubsection{Decision Trees}
\begin{figure}[H]
	\centering
	\includegraphics[width=1\textwidth]{decTree}
	\caption{Example of a Decision Tree \cite{KDNrf}}
	\label{fig:decTree}
\end{figure}

Decision trees are a widespread method for many different machine learning tasks and can be used for either classification or regression.\\
The aim of decision trees is to build a model that predicts the value of a target variable based on different input variables. 

Each node of the tree corresponds to one of the input variables. Each leaf represents a value of the target variable, given the values of the input variables represented by the path from the root to the leaf (fig \ref{fig:decTree}) \cite{wiki:DT}.

The data usually has the following structure: 
\begin{equation}
	(x, Y) = (x_1, x_2, \dots, x_n, Y)
\end{equation}
Where \textit{Y} is the dependent variable that we are trying to classify and \textit{x} is the feature vector.

A decision tree is able to represent a classification problem:
The arcs of a tree's node labeled with input features are labeled with the possible values for that feature, and they can lead to two options:
\begin{itemize}[noitemsep]
	\item A subordinate decision node, with different input features.
	\item A leaf labeled with the resulting class (or a probability).
\end{itemize}

A tree can be learned by dividing the training set into subsets based on the values of the attributes. This is done recursively until a node has all the same value of the target variable, or when splitting is no longer aiding the prediction's result.

Decision trees, however, are often inaccurate. Specifically, by growing trees very deep, they tend to learn highly irregular patterns by overfitting to the training set (they normally have low bias but very high variance) \cite{ESL}. 

Random forests are a way of averaging multiple deep decision trees, trained on different parts of the same training set, with the goal of reducing the variance. \\
This comes at the expense of a small increase in the bias and some loss of interpretability, but generally greatly boosts the performance in the final model \cite{ESL}. 

\subsubsection{Tree Bagging}
Given a set of $n$ independent observations $X_1, \dots, X_n$, each with variance $\sigma^2$, the variance of the mean $\overline{X}$ of the observations is given by $\sigma^2/n$. \\
This means that averaging a set of observations reduces the variance. So to reduce the variance we can take many training sets from the population, build a prediction model using each training set, and average the resulting predictions. 

Basically we could calculate $\hat{f}^1(x), \hat{f}^2(x), \dots, \hat{f}^B(x)$ using B separate training sets, and then average them in order to obtain a single model with lower variance \cite{ISLR}.

This solution is not practical since we normally lack more than one training set. \\
Instead we utilize \textit{bagging}: we can take repeated samples from the training set in order to generate B different bootstrapped training sets. We then train our method on the b\textit{th} bootstrapped training set, and then average the resulting predictions.

\begin{figure}[H]
	\centering
	\includegraphics[width=1\textwidth]{bagging}
	\caption{Bagging Example \cite{bagging}}
	\label{fig:bagging}
\end{figure}

To apply bagging to regression trees specifically, we simply construct B regression trees using B bootstrapped training sets, and average the resulting predictions. These trees are grown deep, and are not pruned. Hence each individual tree has high variance, but low bias. Averaging these B trees reduces the variance \cite{ISLR}.

If we are using Random Forests for classification (like in this thesis), the simplest approach is the following: \\
Given a test observation, we record the class predicted by each of the $B$ trees, and take a majority vote: the final prediction is the most frequent class among the $B$ predictions.

More formally, given a training set $X = x_1, \dots, x_n$ with responses $Y = y_1, \dots, y_n$, bagging $B$ times selects a random sample with replacement of the training set and fits trees to these samples \cite{wiki:randomforest}:

For $b = 1, \dots, B$:
\begin{enumerate}[noitemsep]
	\item Sample, with replacement, $n$ training examples from $X, Y$; call these $X_b, Y_b$.
	\item Train a classification or regression tree $f_b$ on $X_b, Y_b$.
\end{enumerate}

After training, predictions for unseen samples $x'$ can be made by averaging the predictions from all the individual regression trees on $x'$:

\begin{equation}
{{\hat {f}}={\frac {1}{B}}\sum _{b=1}^{B}f_{b}(x')}
\end{equation}

or by using the majority vote in case of classification.

\begin{figure}[H]
	\centering
	\includegraphics[width=1\textwidth]{baggingDT}
	\caption{Bagging leads to different decision trees \cite{baggingDT}}
	\label{fig:baggingDT}
\end{figure}

%todo : maybe remove the rest
Additionally, an estimate of the uncertainty of the prediction can be made as the standard deviation of the predictions from all the individual regression trees on $x'$:

\begin{equation}
	{\sigma ={\sqrt {\frac {\sum_{b=1}^{B}(f_{b}(x')-{\hat {f}})^{2}}{B-1}}}}
\end{equation}

The number of samples/trees, B, is a free parameter. Typically, a few hundred to several thousand trees are used, depending on the size and nature of the training set. \\
An optimal number of trees B can be found using cross-validation, or by observing the out-of-bag error (\ref{oob}): the mean prediction error on each training sample $x_i$, using only the trees that did not have $x_i$ in their bootstrap sample. \\
The training and test error tend to level off after some number of trees have been fit \cite{ISLR}.

\subsubsection{From Bagging to Random Forests}
Random Forests provide an improvement over bagged decision trees by decorrelating the trees. As in bagging, the decision trees are built on a bootstrapped training set. \\
The difference is that every time a split is considered, a random sample of $m$ predictors is chosen as split candidates from the total $p$ predictors. The split can only use one of those $m$ predictors. $\sqrt{m}$ predictors are taken each split, typically with $m \equiv \sqrt{p}$ \cite{ISLR}.

Practically, at each split in the tree, the algorithm can only consider a subset of the predictors (Fig. \ref{fig:RF}). This works because if there is a very strong predictor, that predictor would be used in the majority of the trees, and then all the bagged trees would look alike and would be highly correlated, thus leading to a small reduction in variance.

By forcing each split to consider only $m$ predictors, on average $(p - m)/p$ of splits will not take into account a strong predictor. This, in turn, will decorrelate the trees.

\begin{figure}[H]
	\centering
	\includegraphics[width=1\textwidth]{RF}
	\caption{Random Forest overview}
	\label{fig:RF}
\end{figure}

\subsubsection{Out of bag Error} \label{oob}
There is a very direct way to estimate the test error of a bagged model: \\
on average, we use $2/3$ of the observations to fit a bagged tree. The remaining $1/3$ that are not utilized are called the out-of-bag (OOB) observations. 

We can predict the response for the i\textit{th} observation using each of the trees in which that observation was OOB. This will generate approximately $B/3$ predictions for the i\textit{th} observation.
We can then average these predictions (or use majority voting for classification). 

We can repeat the process for all the $n$ observations and get an average classification error. \\
The resulting OOB error is a valid estimate of the test error for the bagged model, since the response for each observation is predicted using only the trees that were not fit using that observation \cite{ISLR}. 

\subsubsection{Variable Importance}
Bagging has the advantage of giving increased accuracy over using a single tree, at the cost of interpretability of the data. Using bagging we can no longer have easily interpreted diagrams.

With Random Forest we can instead rank the importance of each predictor: \\
given a dataset $D_n = \{X_i, Y_i\}_{i=1}^n$ we fit a RF to the data. While fitting the data, the out-of-bag error is averaged over the forest. 

To get the importance of the j\textit{th} feature, we permute it's values over the training data and recompute the OOB error.

The importance is given by doing the average of the difference between the OOB error before and after this permutation. Then the score is normalized by the standard deviation of the differences.

\begin{figure}[H]
	\centering
	\includegraphics[width=1\textwidth]{RF_imp}
	\caption{Example of variable importance found with Random Forest \cite{rf_imp}.}
	\label{fig:RF_imp}
\end{figure}

\clearpage

\section{SVM} \label{svm}
Support Vector Machines (SVM) are a supervised machine learning algorithm used for both classification and regression.

The main idea is to find the optimal hyperplane for linearly separable data, and then extend this idea to data that are not linearly separable by mapping this data in a new space using a kernel function.

\subsection{Hyperplane}
What is a hyperplane? In \textit{p}-dimension, a hyperplane is a flat affine subspace of $p - 1$ dimension. For example, in two dimensions it's a line, in three dimensions it's a plane (Fig. \ref{fig:hyperplane}), and so on.

\begin{figure}[H]
	\centering
	\includegraphics[width=1\textwidth]{hyperplane}
	\caption{Two and three dimensional space hyperplanes \cite{hyperplaneimg}}
	\label{fig:hyperplane}
\end{figure}

The definition of an hyperplane for two dimensions is:
\begin{equation} \label{eq:hyper1}
\beta_0 + \beta_1 X_1 + \beta_2 X_2 = 0
\end{equation}

Where any $X = (X_1, X_2)^T$ for which equation \ref{eq:hyper1} holds is a point on the hyperplane.

And it can be extended to an hyperplane of p-dimensions:
\begin{equation} \label{eq:hyper2}
\beta_0 + \beta_1 X_1 + \beta_2 X_2 + \dots + \beta_p X_p = 0
\end{equation}

If $X$ does not satisfy equation \ref{eq:hyper2}, but
\begin{equation} \label{eq:hyper3}
\beta_0 + \beta_1 X_1 + \beta_2 X_2 + \dots + \beta_p X_p > 0 \ | < 0
\end{equation}

we know that $X$ lies to one side or the other of the hyperplane. So a hyperplane divides a space in two halves \cite{ISLR}.

It's possible to build a hyperplane that can separate training data based on their class labels, for example by assigning $\{-1, 1\}$ if the result lies on one side or the other of the hyperplane.

\subsection{Maximal Margin Classifier}
If the data can be separated by a hyperplane, it means that there can be an infinite number of separating hyperplanes. \\
To build a classifier it's necessary to chose one between those infinite hyperplanes, such that the classification results are as accurate as possible.

The distance between the hyperplane and the nearest data point from either side is known as the margin.\\
Generally the choice is the \textit{maximal margin hyperplane}, meaning the hyperplane that \textit{maximizes} the margins (has the farthest minimum distance) for the data we are classifying.

In fact the distance of a data point from the hyperplane is a measure of our confidence that the observation was classified correctly \cite{ISLR}.

\begin{figure}[H]
	\centering
	\includegraphics[width=0.5\textwidth]{hyperplane_margin}
	\caption{Different hyperplanes have different margins \cite{svm_monkeylearn}}
\end{figure}

The idea is that a classifier with a large margin on the training set will have a large margin on the test set, so it will classify it correctly \cite{ISLR}.

\subsection{Support Vectors}
Support vectors are the data points that lie closest to the hyperplane (Fig. \ref{fig:suppvec}), they are also the most difficult data points to classify and have direct influence on the location of the hyperplane. In fact, moving the support vectors would change the hyperplane's location.

\begin{figure}[H]
	\centering
	\includegraphics[width=0.7\textwidth]{svm_support_vectors}
	\caption{Example of Support Vectors}
	\label{fig:suppvec}
\end{figure}

So the objective is to choose a hyperplane with the largest possible margin between it and the support vectors, since the larger is the margin, the lower the generalization error of the classifier \cite{ISLR}.

\subsection{Finding the Maximal Margin Classifier}
Finding the maximal margin hyperplane based on a set of $n$ training observations $x_1, \dots x_n \in \mathbb{R}^p$ and with class labels $y_1 \dots y_n \in \{1, -1\}$, translates to an optimization problem:

Maximize the margin $M$:
\begin{equation} \label{eq:svm1}
	\beta_0, \beta_1, \beta_2, \dots, \beta_p, M
\end{equation}

subject to
\begin{equation} \label{eq:svm2}
	\sum_{j=1}^{p}\beta^2_j = 1
\end{equation}

\begin{equation} \label{eq:svm3}
	y_i(\beta_0 + \beta_1 x_{i1} + \beta_2 x_{i2} + \dots + \beta_p x_{ip}) \ge M \quad \forall i = 1, \dots, n
\end{equation}

Equations \ref{eq:svm2} and \ref{eq:svm3} ensure that each observation is on the correct side of the hyperplane, and at least at distance $M$ from the hyperplane. \\
So $M$ represents the margin of our hyperplane, and the optimization problem chooses $\beta_0, \beta_1, \beta_2, \beta_3, \dots, \beta_p$ to maximize M and find the maximal margin hyperplane \cite{ISLR}.

\subsection{Support Vector Classifier}
Unfortunately this hyperplane does not necessarily exists (fig. \ref{fig:svmnotsep}), but we can extend the concept of the Maximal Margin Classifier to find a hyperplane that almost separates the classes by using a \textit{soft} margin. This is actually what an SVM does.

\begin{figure}[H]
	\centering
	\includegraphics[width=0.75\textwidth]{svm_not_separable}
	\caption{Example of Linearly non separable data \cite{svm_not_sep}}
	\label{fig:svmnotsep}
\end{figure}

Even if a separating hyperplane did exists, we might not want to use it: such a classifier would necessarily  classify all of the training data perfectly, and that would lead to sensitivity to changes in data points (fig. \ref{fig:margin_change}). Also, being so sensitive could mean that it's overfitting the data. 

\begin{figure}[H]
	\centering
	\includegraphics[width=1\textwidth]{margin_change}
	\caption{Adding a single data point (from left to right) can change the maximal margin hyperplane very significantly \cite{ISLR}}
	\label{fig:margin_change}
\end{figure}

Instead we consider a classifier that does not separate the data perfectly, but is less prone to change for individual data points and can better classify \textit{most} of the observations.

This classifier is called \textit{soft margin classifier} or \textit{support vector classifier} (Fig. \ref{fig:soft_svm}). \\
In a support vector classifier (SVC) a small subset of the observations are on the wrong side of the margin, or even on the wrong side of the separating hyperplane.

\begin{figure}[H]
	\centering
	\includegraphics[width=0.75\textwidth]{soft_svm}
	\caption{Soft margin classifier. Some data points are allowed to be misclassified to find a larger margin \cite{soft_svm}}
	\label{fig:soft_svm}
\end{figure}

The optimization problem we are solving now is the following: 

Maximize the margin $M$:
\begin{equation} \label{eq:svm4}
\beta_0, \beta_1, \beta_2, \dots, \beta_p, \epsilon_1, \dots, \epsilon_n, M
\end{equation}

subject to
\begin{equation} \label{eq:svm5}
\sum_{j=1}^{p}\beta^2_j = 1
\end{equation}

\begin{equation} \label{eq:svm6}
y_i(\beta_0 + \beta_1 x_{i1} + \beta_2 x_{i2} + \dots + \beta_p x_{ip}) \ge M(1 - \epsilon_i), \quad \epsilon_i \ge 0, \quad \sum_{i=1}^{n}\epsilon_i \le C
\end{equation}

With $C \ge 0$, and $\epsilon_1, \dots, \epsilon_n$ being slack variables to allow data points to be on the wrong side of the margin.

When we solve equations \ref{eq:svm4}-\ref{eq:svm6} we can classify $x^*$ by determining where it lies on the hyperplane:

\begin{equation} \label{eq:svm7}
f(x^*) = \beta_0 + \beta_1 x_{1}^* + \beta_2 x_{2}^* + \dots + \beta_p x_{p}^*
\end{equation}

Based on the value of $\epsilon_i$ we can make some assumptions:
\begin{itemize}[noitemsep]
	\item If $\epsilon_i = 0$ the \textit{i}th observation is on the correct side of the margin.
	\item If $\epsilon_i > 0$ the \textit{i}th observation is on the wrong side of the margin.
	\item If $\epsilon_i > 1$ the \textit{i}th observation is on the wrong side of the hyperplane.
\end{itemize}

$C$ determines the quantity and quality of the violations of the margin and hyperplane that is tolerable by our classifier, and is called \textit{cost}.
\begin{itemize}[noitemsep]
	\item If $C = 0$ then we cannot violate the margin, and $\epsilon_1 = \dots = \epsilon_n = 0$.
	\item If $C > 0$ then the observations $\le C$ can be on the wrong side of the hyperplane.
\end{itemize} 

When $C$ is small the margin is small and we have a classifier that is very fit to the data, with low bias and high variance. \\
When $C$ is big the margin is larger and the classifier is less fit to the data and has high bias and low variance.

The interesting property of this optimization problem is that only the support vectors (observations that are on the margin or that violate the margin) affect the hyperplane, and thus the classifier obtained. \\
All the observations that lie on the correct side of the margin do not affect the support vector classifier.

\subsection{Kernels} \label{kernel}
Sometimes data is not linearly separable (Fig. \ref{fig:svmnotsep}). When that happens, Support Vector Machines work by enlarging the feature space using specific function called kernels, utilized to accommodate a non-linear boundary between the classes.

The solution to equations \ref{eq:svm4} - \ref{eq:svm6} involves only the inner products of the observations, and not the observations themselves \cite{ISLR}.\\
We define the inner product between two \textit{r}-vectors \textit{a} and \textit{b} as $	\langle a,b \rangle = \sum_{i=1}^{r} a_{i}b_{i}$. So the inner product of two observations $x_i$, $x_i'$ is given by:

\begin{equation} \label{eq:svm9}
	\langle x_i, x_i' \rangle = \sum_{j=1}^{p} x_{ij}x_{i'j}
\end{equation}

the linear SVC can then be described as:

\begin{equation} \label{eq:svm10}
	f(x) = \beta_0 + \sum_{i=1}^{n} \alpha_i \langle x, x_i \rangle
\end{equation}

To estimate $\beta_{0}$ and $\alpha_1 \dots, \alpha_n$ all that is required are the $n \choose 2$ inner products $\langle x_i, x_{i'} \rangle$ between all pairs of training observations \cite{ISLR}.

Every time we compute the inner product (\ref{eq:svm9}) in equation \ref{eq:svm10} we could replace it with a generalization of this form:

\begin{equation} \label{eq:svm11}
K(x_i, x_i')
\end{equation}

Where $K$ is a kernel function. A kernel function is used to quantify the similarity between two observations. 

For example a linear kernel of \textit{p} features is:
\begin{equation} \label{eq:svm12}
	K(x_i, x_i') = \sum_{j=1}^{p} x_{ij}x_{i'j}
\end{equation}

or for a polynomial kernel of degree \textit{d}:
\begin{equation} \label{eq:svm13}
	K(x_i, x_i') = (1 + \sum_{j=1}^{p} x_{ij}x_{i'j})^d
\end{equation}

or a radial kernel:
\begin{equation} \label{eq:svm14}
	K(x_i, x_i') = \exp(-\gamma\sum_{j=1}^{p} (x_{ij}, x_{i'j})^2), \quad \gamma \ge 0
\end{equation} 

%todo: add sigmoid kernel
another example is the sigmoid kernel:
\begin{equation} \label{eq:svm15}
	K(x_i, x_i') = \tanh(-k\sum_{j=1}^{p} (x_{ij}, x_{i'j})^2 + \delta)
\end{equation}

We can visualize some of these kernels in fig. \ref{fig:kernels}
\begin{figure}[H]
	\centering
	\includegraphics[width=1\textwidth]{kernels}
	\caption{Visualization of different kinds of kernels \cite{kernels}.}
	\label{fig:kernels}
\end{figure} 

There are many more kernel functions that can be used based on the shape of the data.

The advantage of using a kernel is first of all computational: by using a kernel we just need to compute $K(x_i, x_i')$ for all $n \choose 2$ distinct pairs ($i, i'$).

Graphically, starting with a two dimensional non linearly separable data, this is what happens:
\begin{figure}[H]
	\centering
	\includegraphics[width=0.75\textwidth]{svm_notsep}
	\caption{Linearly non separable data \cite{kernel_trick}.}
\end{figure} 

We map the data from input space \textit{X} into a transformed feature space \textit{H}, by using a non-linear function $\phi: X \rightarrow H$

\begin{figure}[H]
	\centering
	\includegraphics[width=1\textwidth]{kernel_trick}
	\caption{Transformation of the data in a feature space where the instances from the two classes may be linearly separable \cite{kernel_trick}.}
\end{figure} 

\clearpage

