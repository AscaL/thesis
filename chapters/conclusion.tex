%Since the videos in our database were mostly collected from YouTube, certain uncontrollable factors add to the difficulty of their analysis. \\

\section{Final Considerations}
In this thesis we presented a way to discriminate between lies and truths, based on the extraction of Action Units presence and intensity from videos, through the use of an SVM classifier.

The applications for a system that can accurately detect detection are numerous and space in a lot of fields:
\begin{itemize}
	\item In airport security it could be used to detect ill willed people;
	\item An automated deception system for security checks, for example analyzing the response to simple questions;
	\item It could be used by the police force as an aid for interrogation;
	\item In politics it could be adopted by voters to analyze political speeches.
\end{itemize}
Of course there are privacy concerns that need to be answered when starting to utilize a potentially control oriented technology, first of all privacy invasion.

\section{Future Work} \label{fw}
Some possible developments to be applied to this thesis are:
\begin{itemize}
	\item One of the difficulties of this kind of studies is the lack of a big dataset for high stake deception. It would be beneficial to gather such dataset to improve the results, bettering also the quality of the videos;
	\item Extend Action Units extraction to more than one person in the scene. It would be interesting to consider direct interaction with questions and responses.;
	\item Using temporal information for each video to capture the beginning and the end of the interaction. The lie does not necessarily appear in \textit{all} the video, but probably just at some point, even though the majority of the statements are either deceptive or truthful. We were thinking about implementing a LSTM to observe the temporality;
	\item More modules to consider other than the face. For example we could consider the body, and analyze the voice and speech patterns to extract information from the subject.
\end{itemize}